 \documentclass[answers]{exam}
\newif\ifanswers
\answerstrue % comment out to hide answers

\usepackage{lastpage} % Required to determine the last page for the footer
\usepackage{extramarks} % Required for headers and footers
\usepackage[usenames,dvipsnames]{color} % Required for custom colors
\usepackage{graphicx} % Required to insert images
\usepackage{listings} % Required for insertion of code
\usepackage{courier} % Required for the courier font
\usepackage{lipsum} % Used for inserting dummy 'Lorem ipsum' text into the template
\usepackage{enumerate}
\usepackage{subfigure}
\usepackage{booktabs}
\usepackage{amsmath, amsthm, amssymb}
\usepackage{hyperref}
\usepackage{datetime}
\usepackage{minted}
\settimeformat{ampmtime}
\usepackage{algpseudocode}
\usepackage{algorithmicx}
\usepackage[ruled]{algorithm}
\usepackage{tikz-dependency}

\usepackage{tikz}
\usetikzlibrary{positioning,patterns,fit,calc}
% Margins
\topmargin=-0.45in
\evensidemargin=0in
\oddsidemargin=0in
\textwidth=6.5in
\textheight=9.0in
\headsep=0.25in

\linespread{1.1} % Line spacing

% Set up the header and footer
%\pagestyle{fancy}
%\rhead{\hmwkAuthorName} % Top left header
%\lhead{\hmwkClass: \hmwkTitle} % Top center head
%\lfoot{\lastxmark} % Bottom left footer
%\cfoot{} % Bottom center footer
%\rfoot{Page\ \thepage\ of\ \protect\pageref{LastPage}} % Bottom right footer
%\renewcommand\headrulewidth{0.4pt} % Size of the header rule
%\renewcommand\footrulewidth{0.4pt} % Size of the footer rule

\pagestyle{headandfoot}
\runningheadrule{}
\firstpageheader{CS 224n}{Assignment 3}{}
\runningheader{CS 224n} {Assignment 3} {Page \thepage\ of \numpages}
\firstpagefooter{}{}{} \runningfooter{}{}{}

\setlength\parindent{0pt} % Removes all indentation from paragraphs

%----------------------------------------------------------------------------------------
%	CODE INCLUSION CONFIGURATION
%----------------------------------------------------------------------------------------

\definecolor{MyDarkGreen}{rgb}{0.0,0.4,0.0} % This is the color used for comments
\lstloadlanguages{Python} % Load Perl syntax for listings, for a list of other languages supported see: ftp://ftp.tex.ac.uk/tex-archive/macros/latex/contrib/listings/listings.pdf
\lstset{language=Python, % Use Perl in this example
        frame=single, % Single frame around code
        basicstyle=\footnotesize\ttfamily, % Use small true type font
        keywordstyle=[1]\color{Blue}\bf, % Perl functions bold and blue
        keywordstyle=[2]\color{Purple}, % Perl function arguments purple
        keywordstyle=[3]\color{Blue}\underbar, % Custom functions underlined and blue
        identifierstyle=, % Nothing special about identifiers
        commentstyle=\usefont{T1}{pcr}{m}{sl}\color{MyDarkGreen}\small, % Comments small dark green courier font
        stringstyle=\color{Purple}, % Strings are purple
        showstringspaces=false, % Don't put marks in string spaces
        tabsize=5, % 5 spaces per tab
        %
        % Put standard Perl functions not included in the default language here
        morekeywords={rand},
        %
        % Put Perl function parameters here
        morekeywords=[2]{on, off, interp},
        %
        % Put user defined functions here
        morekeywords=[3]{test},
       	%
        morecomment=[l][\color{Blue}]{...}, % Line continuation (...) like blue comment
        numbers=left, % Line numbers on left
        firstnumber=1, % Line numbers start with line 1
        numberstyle=\tiny\color{Blue}, % Line numbers are blue and small
        stepnumber=5 % Line numbers go in steps of 5
}

%----------------------------------------------------------------------------------------
%	NAME AND CLASS SECTION
%----------------------------------------------------------------------------------------

\newcommand{\hmwkTitle}{Dependency Parsing} % Assignment title
\newcommand{\hmwkClass}{CS\ 224n Assignment \#3} % Course/class
\newcommand{\ifans}[1]{\ifanswers \color{red} \textbf{Solution: } #1 \color{black} \fi}

% \newcommand{\ifans}[1]{}

\input macros.tex
\input std_macros.tex

%----------------------------------------------------------------------------------------
%	TITLE PAGE
%----------------------------------------------------------------------------------------
\qformat{\Large\bfseries\thequestion{}. \thequestiontitle{} (\thepoints{})\hfill}

\title{
\vspace{-1in}
\textmd{\textbf{\hmwkClass:\ \hmwkTitle}}
}
\author{}
%\date{\textit{\small Updated \today\ at \currenttime}} % Insert date here if you want it to appear below your name
\date{}

\setcounter{section}{0} % one-indexing
\begin{document}

\maketitle
% \vspace{-10pt}

In this assignment, you will build a neural dependency parser using PyTorch. For a review of the fundamentals of PyTorch, please check out the PyTorch review session on Canvas.  In Part 1, you will learn about two general neural network techniques (Adam Optimization and Dropout). In Part 2, you will implement and train a dependency parser using the techniques from Part 1, before analyzing a few erroneous dependency parses.

Please tag the questions correctly on Gradescope, the TAs  will take points off if you don't tag questions. 
\begin{questions}
    \titledquestion{Machine Learning \& Neural Networks}[8] 
\begin{parts}

    
    \part[4] Adam Optimizer\newline
        Recall the standard Stochastic Gradient Descent update rule:
        \alns{
            	\btheta_{t+1} &\gets \btheta_t - \alpha \nabla_{\btheta_t} J_{\text{minibatch}}(\btheta_t)
        }
        where $t+1$ is the current timestep, $\btheta$ is a vector containing all of the model parameters, ($\btheta_t$ is the model parameter at time step $t$, and $\btheta_{t+1}$ is the model parameter at time step $t+1$), $J$ is the loss function, $\nabla_{\btheta} J_{\text{minibatch}}(\btheta)$ is the gradient of the loss function with respect to the parameters on a minibatch of data, and $\alpha$ is the learning rate.
        Adam Optimization\footnote{Kingma and Ba, 2015, \url{https://arxiv.org/pdf/1412.6980.pdf}} uses a more sophisticated update rule with two additional steps.\footnote{The actual Adam update uses a few additional tricks that are less important, but we won't worry about them here. If you want to learn more about it, you can take a look at: \url{http://cs231n.github.io/neural-networks-3/\#sgd}}
            
        \begin{subparts}

            \subpart[2]First, Adam uses a trick called {\it momentum} by keeping track of $\bm$, a rolling average of the gradients:
                \alns{
                	\bm_{t+1} &\gets \beta_1\bm_{t} + (1 - \beta_1)\nabla_{\btheta_t} J_{\text{minibatch}}(\btheta_t) \\
                	\btheta_{t+1} &\gets \btheta_t - \alpha \bm_{t+1}
                }
                where $\beta_1$ is a hyperparameter between 0 and 1 (often set to  0.9). Briefly explain in 2--4 sentences (you don't need to prove mathematically, just give an intuition) how using $\bm$ stops the updates from varying as much and why this low variance may be helpful to learning, overall.\newline
                
                \ifans{since $\beta_1$ is a hyper parameter taking the values between $0$ and $1$ and $\beta_1$ is usually in the higher range it only updates a small subset of values in the proportion $1-\beta_1$. This helps us maintain a small variance and reach a local minimum. Updating only the values in $1-\beta_1$ helps prevent the model parameters from moving around too much when moving towards local minimum. at every iteration there is an exponentially decaying average of negative weight gradients which causes the update step not to be instantaneous but rather depend by some amount on previous updates. So momentum prefers flat minima which helps to generalize better.}
              
           
            \subpart[2] Adam extends the idea of {\it momentum} with the trick of {\it adaptive learning rates} by keeping track of  $\bv$, a rolling average of the magnitudes of the gradients:
                \alns{
                	\bm_{t+1} &\gets \beta_1\bm_{t} + (1 - \beta_1)\nabla_{\btheta_t} J_{\text{minibatch}}(\btheta_t) \\
                	\bv_{t+1} &\gets \beta_2\bv_{t} + (1 - \beta_2) (\nabla_{\btheta_t} J_{\text{minibatch}}(\btheta_t) \odot \nabla_{\btheta_t} J_{\text{minibatch}}(\btheta_t)) \\
                	\btheta_{t+1} &\gets \btheta_t - \alpha \bm_{t+1} / \sqrt{\bv_{t+1}}
                }
                where $\odot$ and $/$ denote element-wise multiplication and division (so $\bz \odot \bz$ is elementwise squaring) and $\beta_2$ is a hyper parameter between 0 and 1 (often set to  0.99). Since Adam divides the update by $\sqrt{\bv}$, which of the model parameters will get larger updates?  Why might this help with learning?
                
                \ifans{The parameters with the smallest gradients will get the larger updates because, if an accumulated square norm is very small, then dividing learning rate by what is small will cause larger values in corresponding gradient axes, thus larger step sizes.. This means that are at a place where the loss with respect to them is pretty flat will get larger updates, helping them move off the flat areas.}
           
                
                \end{subparts}
        
        
            \part[4] 
            Dropout\footnote{Srivastava et al., 2014, \url{https://www.cs.toronto.edu/~hinton/absps/JMLRdropout.pdf}} is a regularization technique. During training, dropout randomly sets units in the hidden layer $\bh$ to zero with probability $p_{\text{drop}}$ (dropping different units each minibatch), and then multiplies $\bh$ by a constant $\gamma$. We can write this as:
                \alns{
                	\bh_{\text{drop}} = \gamma \bd \odot \bh
                }
                where $\bd \in \{0, 1\}^{D_h}$ ($D_h$ is the size of $\bh$)
                is a mask vector where each entry is 0 with probability $p_{\text{drop}}$ and 1 with probability $(1 - p_{\text{drop}})$. $\gamma$ is chosen such that the expected value of $\bh_{\text{drop}}$ is $\bh$:
                \alns{
                	\mathbb{E}_{p_{\text{drop}}}[\bh_\text{drop}]_i = h_i \text{\phantom{aaaa}}
                }
                for all $i \in \{1,\dots,D_h\}$. 
            \begin{subparts}
            \subpart[2]
                What must $\gamma$ equal in terms of $p_{\text{drop}}$? Briefly justify your answer or show your math derivation using the equations given above.
                
                \ifans{We need to scale the output by $\gamma$ to ensure that the scale of the outputs at test time is identical to the expected outputs at training time. If we dont multiply by $\gamma$ we can see that during training the expected value of any output is: $$E_{p_{drop}}[h_{drop}]_{i} = (1-p_{drop}h_i)$$ During testing when nothing is dropped we would need to multiply the output vector by $(1-p_{drop})$ to match the expectation during training. To keep testing unchanged we would need to apply inverse dropout and multiply the output during training by the inverse of the value. $$\gamma = \frac{1}{1-p_{drop}}$$}
                
            
          \subpart[2] Why should dropout be applied during training? Why should dropout \textbf{NOT} be applied during evaluation? (Hint: it may help to look at the paper linked above in the write-up.) \newline
          \ifans{Dropout is a regularization technique that minimizes the chances of the neural network being too dependent on one particular feature for prediction and also prevents over-fitting. It does this by dropping out some of the input values to 0. This way the network learns to be not dependent on specific nodes for prediction and learns to average the many nodes. \newline
          	We do not want to do this during evaluation as we want to evaluate the neural network at full capacity and we want the neural network to depend on all the nodes for predictions during testing. We want to use all the information from the trained neurons and thus we do not use dropout technique when we are evaluating. 
          }
         
        \end{subparts}


\end{parts}

    \newpage
    \graphicspath{ {images/} }

\titledquestion{Analyzing NMT Systems}[25]

\begin{parts}

    \part[3] Look at the {\monofam{src.vocab}} file for some examples of phrases and words in the source language vocabulary. When encoding an input Mandarin Chinese sequence into ``pieces'' in the vocabulary, the tokenizer maps the sequence to a series of vocabulary items, each consisting of one or more characters (thanks to the {\monofam{sentencepiece}} tokenizer, we can perform this segmentation even when the original text has no white space). Given this information, how could adding a 1D Convolutional layer after the embedding layer and before passing the embeddings into the bidirectional encoder help our NMT system? \textbf{Hint:} each Mandarin Chinese character is either an entire word or a morpheme in a word. Look up the meanings of 电, 脑, and 电脑 separately for an example. The characters 电 (electricity) and  脑 (brain) when combined into the phrase 电脑 mean computer.

    \ifans{}


    \part[8] Here we present a series of errors we found in the outputs of our NMT model (which is the same as the one you just trained). For each example of a reference (i.e., `gold') English translation, and NMT (i.e., `model') English translation, please:
    
    \begin{enumerate}
        \item Identify the error in the NMT translation.
        \item Provide possible reason(s) why the model may have made the error (either due to a specific linguistic construct or a specific model limitation).
        \item Describe one possible way we might alter the NMT system to fix the observed error. There are more than one possible fixes for an error. For example, it could be tweaking the size of the hidden layers or changing the attention mechanism.
    \end{enumerate}
    
    Below are the translations that you should analyze as described above. Only analyze the underlined error in each sentence. Rest assured that you don't need to know Mandarin to answer these questions. You just need to know English! If, however, you would like some additional color on the source sentences, feel free to use a resource like \url{https://www.archchinese.com/chinese_english_dictionary.html} to look up words. Feel free to search the training data file to have a better sense of how often certain characters occur.

    \begin{subparts}
        \subpart[2]
        \textbf{Source Sentence:} 贼人其后被警方拘捕及被判处盗窃罪名成立。 \newline
        \textbf{Reference Translation:} \textit{\underline{the culprits were} subsequently arrested and convicted.}\newline
        \textbf{NMT Translation:} \textit{\underline{the culprit was} subsequently arrested and sentenced to theft.}
        
        \ifans{}


        \subpart[2]
        \textbf{Source Sentence}: 几乎已经没有地方容纳这些人,资源已经用尽。\newline
        \textbf{Reference Translation}: \textit{there is almost no space to accommodate these people, and resources have run out.   }\newline
        \textbf{NMT Translation}: \textit{the resources have been exhausted and \underline{resources have been exhausted}.}
        
        \ifans{}

        \subpart[2]
        \textbf{Source Sentence}: 当局已经宣布今天是国殇日。 \newline
        \textbf{Reference Translation}: \textit{authorities have announced \underline{a national mourning today.}}\newline
        \textbf{NMT Translation}: \textit{the administration has announced \underline{today's day.}}
        
        \ifans{}
        
        \subpart[2] 
        \textbf{Source Sentence\footnote{This is a Cantonese sentence! The data used in this assignment comes from GALE Phase 3, which is a compilation of news written in simplified Chinese from various sources scraped from the internet along with their translations. For more details, see \url{https://catalog.ldc.upenn.edu/LDC2017T02}. }:} 俗语有云:``唔做唔错"。\newline
        \textbf{Reference Translation:} \textit{\underline{`` act not, err not "}, so a saying goes.}\newline
        \textbf{NMT Translation:} \textit{as the saying goes, \underline{`` it's not wrong. "}}
        
        \ifans{}
    \end{subparts}


    \part[14] BLEU score is the most commonly used automatic evaluation metric for NMT systems. It is usually calculated across the entire test set, but here we will consider BLEU defined for a single example.\footnote{This definition of sentence-level BLEU score matches the \texttt{sentence\_bleu()} function in the \texttt{nltk} Python package. Note that the NLTK function is sensitive to capitalization. In this question, all text is lowercased, so capitalization is irrelevant. \\ \url{http://www.nltk.org/api/nltk.translate.html\#nltk.translate.bleu_score.sentence_bleu}
    } 
    Suppose we have a source sentence $\bs$, a set of $k$ reference translations $\br_1,\dots,\br_k$, and a candidate translation $\bc$. To compute the BLEU score of $\bc$, we first compute the \textit{modified $n$-gram precision} $p_n$ of $\bc$, for each of $n=1,2,3,4$, where $n$ is the $n$ in \href{https://en.wikipedia.org/wiki/N-gram}{n-gram}:
    \begin{align}
        p_n = \frac{ \displaystyle \sum_{\text{ngram} \in \bc} \min \bigg( \max_{i=1,\dots,k} \text{Count}_{\br_i}(\text{ngram}), \enspace \text{Count}_{\bc}(\text{ngram}) \bigg) }{\displaystyle \sum_{\text{ngram}\in \bc} \text{Count}_{\bc}(\text{ngram})}
    \end{align}
     Here, for each of the $n$-grams that appear in the candidate translation $\bc$, we count the maximum number of times it appears in any one reference translation, capped by the number of times it appears in $\bc$ (this is the numerator). We divide this by the number of $n$-grams in $\bc$ (denominator). \newline 

    Next, we compute the \textit{brevity penalty} BP. Let $len(c)$ be the length of $\bc$ and let $len(r)$ be the length of the reference translation that is closest to $len(c)$ (in the case of two equally-close reference translation lengths, choose $len(r)$ as the shorter one). 
    \begin{align}
        BP = 
        \begin{cases}
            1 & \text{if } len(c) \ge len(r) \\
            \exp \big( 1 - \frac{len(r)}{len(c)} \big) & \text{otherwise}
        \end{cases}
    \end{align}
    Lastly, the BLEU score for candidate $\bc$ with respect to $\br_1,\dots,\br_k$ is:
    \begin{align}
        BLEU = BP \times \exp \Big( \sum_{n=1}^4 \lambda_n \log p_n \Big)
    \end{align}
    where $\lambda_1,\lambda_2,\lambda_3,\lambda_4$ are weights that sum to 1. The $\log$ here is natural log.
    \newline
    \begin{subparts}
        \subpart[5] Please consider this example: \newline
        Source Sentence $\bs$: \textbf{需要有充足和可预测的资源。} 
        \newline
        Reference Translation $\br_1$: \textit{resources have to be sufficient and they have to be predictable}
        \newline
        Reference Translation $\br_2$: \textit{adequate and predictable resources are required}
        
        NMT Translation $\bc_1$: there is a need for adequate and predictable resources
        
        NMT Translation $\bc_2$: resources be sufficient and predictable to
        
        Please compute the BLEU scores for $\bc_1$ and $\bc_2$. Let $\lambda_i=0.5$ for $i\in\{1,2\}$ and $\lambda_i=0$ for $i\in\{3,4\}$ (\textbf{this means we ignore 3-grams and 4-grams}, i.e., don't compute $p_3$ or $p_4$). When computing BLEU scores, show your work (i.e., show your computed values for $p_1$, $p_2$, $len(c)$, $len(r)$ and $BP$). Note that the BLEU scores can be expressed between 0 and 1 or between 0 and 100. The code is using the 0 to 100 scale while in this question we are using the \textbf{0 to 1} scale. Please round your responses to 3 decimal places. 
        \newline
        
        Which of the two NMT translations is considered the better translation according to the BLEU Score? Do you agree that it is the better translation?
        
        \ifans{}
        
        \subpart[5] Our hard drive was corrupted and we lost Reference Translation $\br_1$. Please recompute BLEU scores for $\bc_1$ and $\bc_2$, this time with respect to $\br_2$ only. Which of the two NMT translations now receives the higher BLEU score? Do you agree that it is the better translation?
        
        \ifans{}
        
        \subpart[2] Due to data availability, NMT systems are often evaluated with respect to only a single reference translation. Please explain (in a few sentences) why this may be problematic. In your explanation, discuss how the BLEU score metric assesses the quality of NMT translations when there are multiple reference transitions versus a single reference translation.
        
        \ifans{}
        
        \subpart[2] List two advantages and two disadvantages of BLEU, compared to human evaluation, as an evaluation metric for Machine Translation. 
        
        \ifans{}
        
    \end{subparts}
\end{parts}

\end{questions}

\Large{\textbf{Submission Instructions}}

\normalsize
You shall submit this assignment on GradeScope as two submissions -- one for ``Assignment 4 [coding]" and another for `Assignment 4 [written]":
\begin{enumerate}
    \item Run the \texttt{collect\_submission.sh} script on Azure to produce your \texttt{assignment4.zip} file. You can use \href{http://www.hypexr.org/linux_scp_help.php}{scp} to transfer files between Azure and your local computer.
    \item Upload your \texttt{assignment4.zip} file to GradeScope to ``Assignment 4 [coding]".
    \item Upload your written solutions to GradeScope to ``Assignment 4 [written]". When you submit your assignment, make sure to tag all the pages for each problem according to Gradescope's submission directions. Points will be deducted if the submission is not correctly tagged.
\end{enumerate}
\end{document}